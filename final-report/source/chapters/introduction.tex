\section{Midterm-report}

\subsection{Project Recap}
This project investigates the architecture of a gene interaction network obtained from the BIO-CE-GN dataset for genetic interactions in Caenorhabditis elegans.

The original proposal sought to explore the network using graph-theoretical techniques to extract core genes using different centrality methods and to explore potential community structures.

The project uses network topology to elucidate underlying regulators and organisational relationships in the biological system by modelling genes as nodes and interactions as edges.

\section{Progress Since the Proposal}

\subsection{Dataset Processing and Network Construction}
During the development phase, the \textbf{BIO-CE-GN dataset} has been processed and adapted as a novel gene interaction network for graph analysis.

The self-loops were removed from the graph and the data are represented as a simple undirected graph to ensure consistency in centrality computation.The network that comes out the product of these steps is composed of 2,220 nodes and 53,683 edges, with an average degree of 48.36 and a density of 0.0218.The graph consists of three connected components.

\subsection{Centrality Analysis and Preliminary Results}
After construction of the network, various node-level centrality measures including degree centrality, betweenness centrality, PageRank were computed.All node centrality values were calculated and exported for further evaluation.

Further, Pearson correlation matrix between the centrality measures was calculated and displayed by using a correlation heatmap.

The preliminary results suggest a strong relationship between degree centrality with PageRank, and betweenness centrality shows relatively low correlation with the other measures, which indicates that it captures complementary structural information.

\section{Methodology}

\subsection{Problem Definition}

The primary goal is to characterise the connectivity and community structure of the gene interaction network derived from \textit{C. elegans}. 
We aim to identify critical genes (hubs or regulators), analyse modularity, and study how topological patterns relate to biological functionality.

\subsection{Data Preprocessing}

Before performing any analysis, we cleaned and formatted the dataset to ensure compatibility with network analysis libraries. 
This involved:
\begin{itemize}
    \item Removing self-loops and duplicated edges;
    \item Handling missing or inconsistent identifiers;
    \item Converting the data into an \textbf{edge list} format suitable for the \texttt{networkx} Python library.
\end{itemize}

In addition, we removed self-loops and converted the dataset into a simple undirected graph in order to compute centrality measures consistently.  
All preprocessing steps were performed in Python using the NetworkX library.

\subsection{Graph Representation}

The dataset was represented as an undirected graph \( G = (V, E) \), where:
\begin{itemize}
    \item \( V \) represents the set of genes;
    \item \( E \) represents the genetic interactions between them.
\end{itemize}

The graph was constructed using the \texttt{networkx} library, which provides efficient tools for computing network metrics such as centrality, clustering coefficients, and community structures.

\subsection{Algorithms Used}

We computed several network metrics to analyse different aspects of the network:
\begin{itemize}
    \item \textbf{Degree Centrality:} Measures the number of connections each gene has, identifying highly connected hub genes.
    \item \textbf{Betweenness Centrality:} Quantifies how often a gene lies on the shortest paths between other genes, identifying key regulatory genes.
    \item \textbf{PageRank:} Evaluates the relative importance of genes based on their connectivity to other important genes.
    \item \textbf{Clustering Coefficient:} Determines the tendency of genes to form tightly connected groups, representing potential biological modules.
    \item \textbf{Average Neighbor Degree:} Indicates how the connectivity of a gene correlates with that of its neighbours.
\end{itemize}

\subsection{Intended Experiments}

In the following experiments, we will examine the BIO-CE-GN dataset to describe its main topological and biological characteristics.

\begin{itemize}
    \item \textbf{Graph preprocessing} — Represent the dataset as an undirected gene–gene interaction network after removing self-loops and duplicated edges.
    \item \textbf{Network properties} — Compute basic network metrics such as node and edge counts, degree distribution, and connectivity. \\
     After preprocessing, the network contains 2,220 nodes and 53,683 edges, 
     with an average degree of 48.36 and a density of 0.0218. 
     We also computed centrality scores for all nodes and generated the Pearson 
     correlation matrix between degree, betweenness and PageRank.
    \item \textbf{Centrality and correlation} — Calculate degree, closeness, and betweenness centralities, and analyze their correlations to identify genes with high centrality.  
    We exported the top-ranked nodes for each centrality measure and generated summary plots.  
    These results will be used in the next stage of the project for biological interpretation and pathway validation.
    \item \textbf{Clustering relevance} — Explore communities and assess whether clustered nodes are involved in known biological pathways.  
    \item \textbf{Visualization and scalability} — Visualize main structures and test algorithm performance on larger datasets.
\end{itemize}

Analysis for all the above steps will be performed using \textbf{Python}, \textbf{NetworkX}, and \textbf{Jupyter Notebook} on local machines.
