\section{Introduction}

\subsection{Motivation}
How are genes interconnected within a biological system? 
Which genes play the most central roles in maintaining the organisation and functionality of the network? 
And do these interactions form cohesive functional modules or rather a loosely connected structure?

Understanding these questions is fundamental in biology, as the structure of gene and protein interaction networks often reflects the organisation of the underlying cellular processes.
By representing biological systems as graphs — where nodes correspond to genes and edges represent genetic interactions — we can apply concepts from graph theory to uncover hidden patterns of regulation and connectivity.

In particular, analysing centrality measures allows us to identify hub genes that may act as key regulators, while studying clustering and community structures reveals how these genes group into functional modules. 

The BIO-CE-GN dataset, representing genetic interactions in \textit{Caenorhabditis elegans}, provides an ideal test case for such an analysis. 
Through a combination of centrality and clustering analyses, this project aims to characterise the topology of the network and answer these fundamental questions about its organisation and functional architecture.

\subsection{Objectives}
The main objective of this project is to analyse the structure of the BIO-CE-GN gene interaction network by applying several graph-theoretical measures. 
Specifically, we aim to:
\begin{itemize}
    \item Identify central genes using centrality metrics such as \textbf{degree}, \textbf{betweenness}, and \textbf{PageRank};
    \item Evaluate the \textbf{clustering} and \textbf{community structure} to detect functional gene modules;
    \item Compare different centrality measures to assess their biological interpretability and computational scalability.
\end{itemize}

\subsection{Dataset Overview}
The \textbf{BIO-CE-GN dataset} consists of experimentally derived genetic interactions among genes of \textit{Caenorhabditis elegans}, a well-studied model organism. 
Each node represents a gene, and each edge denotes an experimentally verified genetic interaction. 
The dataset provides both directional and weighted information, indicating the type and strength of interactions.  

The network contains approximately \textit{N} nodes and \textit{M} edges (exact values to be confirmed), forming a dense and biologically meaningful structure. 
This dataset enables the exploration of complex topological properties that may correspond to real biological pathways and regulatory mechanisms.

\section{Methodology}

\subsection{Problem Definition}

The primary goal is to characterise the connectivity and community structure of the gene interaction network derived from \textit{C. elegans}. 
We aim to identify critical genes (hubs or regulators), analyse modularity, and study how topological patterns relate to biological functionality.

\subsection{Data Preprocessing}

Before performing any analysis, we cleaned and formatted the dataset to ensure compatibility with network analysis libraries. 
This involved:
\begin{itemize}
    \item Removing self-loops and duplicated edges;
    \item Handling missing or inconsistent identifiers;
    \item Converting the data into an \textbf{edge list} format suitable for the \texttt{networkx} Python library.
\end{itemize}

\subsection{Graph Representation}

The dataset was represented as an undirected graph \( G = (V, E) \), where:
\begin{itemize}
    \item \( V \) represents the set of genes;
    \item \( E \) represents the genetic interactions between them.
\end{itemize}

The graph was constructed using the \texttt{networkx} library, which provides efficient tools for computing network metrics such as centrality, clustering coefficients, and community structures.

\subsection{Algorithms Used}

We computed several network metrics to analyse different aspects of the network:
\begin{itemize}
    \item \textbf{Degree Centrality:} Measures the number of connections each gene has, identifying highly connected hub genes.
    \item \textbf{Betweenness Centrality:} Quantifies how often a gene lies on the shortest paths between other genes, identifying key regulatory genes.
    \item \textbf{PageRank:} Evaluates the relative importance of genes based on their connectivity to other important genes.
    \item \textbf{Clustering Coefficient:} Determines the tendency of genes to form tightly connected groups, representing potential biological modules.
    \item \textbf{Average Neighbor Degree:} Indicates how the connectivity of a gene correlates with that of its neighbours.
\end{itemize}

\subsection{Intended Experiments}

We plan to:
\begin{itemize}
    \item Analyse correlations between different centrality measures;
    \item Visualise the network to highlight highly central genes;
    \item Investigate whether topologically clustered genes correspond to known biological pathways;
    \item Explore scalability of algorithms when applied to large-scale biological datasets.
\end{itemize}
