\section{Introduction}

\subsection{Motivation}
How are genes interconnected within a biological system? 
Which genes play the most central roles in maintaining the organisation and functionality of the network? 
And do these interactions form cohesive functional modules or rather a loosely connected structure?

Understanding these questions is fundamental in biology, as the structure of gene and protein interaction networks often reflects the organisation of the underlying cellular processes.
By representing biological systems as graphs — where nodes correspond to genes and edges represent genetic interactions — we can apply concepts from graph theory to uncover hidden patterns of regulation and connectivity.

In particular, analysing centrality measures allows us to identify hub genes that may act as key regulators, while studying clustering and community structures reveals how these genes group into functional modules. 

The BIO-CE-GN dataset, representing genetic interactions in \textit{Caenorhabditis elegans}, provides an ideal test case for such an analysis. 
Through a combination of centrality and clustering analyses, this project aims to characterise the topology of the network and answer these fundamental questions about its organisation and functional architecture.

\subsection{Objectives}

\subsection{Dataset Overview}
