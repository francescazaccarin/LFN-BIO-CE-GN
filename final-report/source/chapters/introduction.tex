\section{Midterm-report}

\subsection{Project Recap}
This project investigates the architecture of a gene interaction network obtained from the BIO-CE-GN dataset for genetic interactions in Caenorhabditis elegans.

The original proposal sought to explore the network using graph-theoretical techniques to extract core genes using different centrality methods and to explore potential community structures.

The project uses network topology to elucidate underlying regulators and organisational relationships in the biological system by modelling genes as nodes and interactions as edges.

\section{Progress Since the Proposal}

\subsection{Dataset Processing and Network Construction}
During the development phase, the \textbf{BIO-CE-GN dataset} has been processed and adapted as a novel gene interaction network for graph analysis.

The self-loops were removed from the graph and the data are represented as a simple undirected graph to ensure consistency in centrality computation.The network that comes out the product of these steps is composed of 2,220 nodes and 53,683 edges, with an average degree of 48.36 and a density of 0.0218.The graph consists of three connected components.

\subsection{Centrality Analysis and Preliminary Results}
After construction of the network, various node-level centrality measures including degree centrality, betweenness centrality, PageRank were computed.All node centrality values were calculated and exported for further evaluation.

Further, Pearson correlation matrix between the centrality measures was calculated and displayed by using a correlation heatmap.

The preliminary results suggest a strong relationship between degree centrality with PageRank, and betweenness centrality shows relatively low correlation with the other measures, which indicates that it captures complementary structural information.

\section{Interpretation of Current Results}

The node-level analysis based on different centrality measures provides complementary insights into the structural roles of genes in the BIO-CE-GN interaction network. In particular, degree centrality, betweenness centrality, and PageRank capture different notions of node importance within the network.

As discussed in class, degree centrality reflects local connectivity by measuring how many direct interactions a gene has, while betweenness centrality highlights genes that act as bridges between different parts of the network. PageRank, instead, captures a notion of global importance by accounting not only for the number of connections, but also for the importance of neighbouring nodes from a random-walk perspective.

The correlation analysis further clarifies the relationship between these measures. The strong correlation observed between degree centrality and PageRank suggests that genes with many interactions also tend to be globally influential within the network. In contrast, betweenness centrality exhibits lower correlation with the other measures, indicating that it captures complementary structural information by identifying genes that play a mediating or connector role. These differences motivate a combined use of centrality measures to characterise gene importance from multiple structural perspectives.

\section{Changes from the Original Proposal}

Some changes have been reported in the projects from planning. Although the main concerns of such changes are the options chosen to model the data, the nature of the midterm analysis, and the scheduling of the remaining tasks.

First, as shown in the proposal, the BIO-CE-GN dataset was already described at a high level, but the dataset is fully processed and shown as a graph-based gene interaction network. Specifically, we represented the network as an undirected graph with no self-loops. This model was chosen as it guarantees reproducibility and comparability among the centrality measures.

Second, the midterm phase is now focused on conducting a detailed structural analysis at the node level. Furthermore, apart from calculating degree centrality, betweenness centrality and PageRank (as initially planned to) it is also found out that there can be correlation analysis done and visualized between the measures. This step was not mentioned in the original proposal and is another insight into whether various notions of node importance are necessarily conjoined.

Finally, elements of the proposal such as detailed biological interpretation of central genes and community detection analyses were not original and are scheduled later in the project. This enables the present work to consolidate the network-level results and continue on to biological validation and at the further functional interpretation level.

\section{Next Steps}

The second phase of the project will concentrate on generalizing the network-level results to biological understanding. Especially the most central genes determined by the centrality analysis will be investigated to understand their biological importance and functions within the genetic interaction network.

Community detection and clustering analyses will be utilized to explore if densely associated groups of genes map onto known biological pathways or functional modules. This task was described in the original experimental plan and will be addressed in the subsequent stage of the project.

These findings will serve as a basis for additional validation and refinement of the analysis so that in the final report, the network architecture and its biological implications can be understood more completely.
