\newpage
\section*{Appendix — References, Contributions and Use of AI Tools}

\vspace{0.5cm}

\textbf{Reference.} 

\begin{itemize}
    \item Rossi, R. A., \& Ahmed, N. K. (2015). \textit{The Network Data Repository with Interactive Graph Analytics and Visualization.} AAAI. \url{https://networkrepository.com}
    \item Newman, M. E. J. (2003). \textit{The Structure and Function of Complex Networks.} SIAM Review, 45(2), 167–256.
    \item Brandes, U. (2001). \textit{A Faster Algorithm for Betweenness Centrality.} Journal of Mathematical Sociology, 25(2), 163–177.
    \item Watts, D. J., \& Strogatz, S. H. (1998). \textit{Collective Dynamics of ‘Small-World’ Networks.} Nature, 393(6684), 440–442.
\end{itemize}

\vspace{0.5cm}

\textbf{Contributions.}  

\begin{itemize}
    \item \textbf{Francesca Zaccarin (2156564):} 
    Development of the section related to the BIO-CE-GN dataset, including the biological motivation, objectives, and methodological design. 
    She also managed the configuration of the LaTeX project structure on GitHub and the integration of automated PDF compilation workflows.
    
    \item \textbf{Shen Zhang (2157248):} 
    Wrote and structured the Motivation, Intended Experiments, and Appendix sections. Also contributed to revising the overall proposal structure, reference organization, and final document polishing.
    
\end{itemize}

\vspace{0.5cm}

\textbf{Use of AI Tools.}  

ChatGPT was used as a technical assistant to configure and debug the \texttt{latex.yml} workflow for GitHub Actions, which was previously unfamiliar. 
It was also employed to refine the structure of the LaTeX source files and correct minor syntax issues. 
No automatic code generation, data analysis, or content writing was delegated to AI tools; all project concepts, structure, and implementation were developed by the authors.

