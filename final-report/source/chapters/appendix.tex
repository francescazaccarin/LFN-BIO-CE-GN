\newpage
\section*{Appendix — References, Contributions and Use of AI Tools}

\vspace{0.5cm}

\textbf{Reference.} 

\begin{itemize}
    \item Rossi, R. A., \& Ahmed, N. K. (2015). \textit{The Network Data Repository with Interactive Graph Analytics and Visualization.} AAAI. \url{https://networkrepository.com}
    \item Newman, M. E. J. (2003). \textit{The Structure and Function of Complex Networks.} SIAM Review, 45(2), 167–256.
    \item Brandes, U. (2001). \textit{A Faster Algorithm for Betweenness Centrality.} Journal of Mathematical Sociology, 25(2), 163–177.
    \item Watts, D. J., \& Strogatz, S. H. (1998). \textit{Collective Dynamics of ‘Small-World’ Networks.} Nature, 393(6684), 440–442.
\end{itemize}

\vspace{0.5cm}

\textbf{Contributions.}  

\begin{itemize}
   \item \textbf{Francesca Zaccarin (2156564):} 
    Led the technical development of the project by loading, preprocessing, and constructing the BIO-CE-GN gene interaction network.
    
    She implemented the data cleaning procedures, including the removal of self-loops and the conversion of the dataset into an undirected graph suitable for network analysis.
    Francesca computed the main network statistics and node-level centrality measures, including degree centrality, betweenness centrality, and PageRank, and performed the correlation analysis between these measures.
    She also generated and exported all relevant results and figures, including centrality tables and correlation heatmaps.
    In addition, Francesca contributed to drafting and revising sections related to the dataset description and methodological setup, and managed the configuration of the LaTeX project structure on GitHub, including automated PDF compilation workflows.

    
    \item \textbf{Shen Zhang (2157248):} 
    Focused on the analytical interpretation, organisation, and writing of the report.
    
    Shen structured and wrote the midterm report sections describing project progress, interpretation of current results, modifications from the original proposal, and planned next steps.
    She was responsible for interpreting the centrality and correlation results from a network-structural and biological perspective, ensuring consistency between computational outputs and written analysis.
    Shen also revised and consolidated the overall document structure, refined the logical flow between sections, and organised the appendix materials, including references and contribution statements.
    
\end{itemize}

\vspace{0.5cm}

\textbf{Use of AI Tools.}

In this report, AI tools were helpful assistants during the preparation. Specifically, ChatGPT was used to assist with clarifying conceptual questions related to network analysis, refining the form and clarity of written sections, and assisting in language polish and formatting in LaTeX.

The authors handled the entire process of the data processing, network construction, computational analysis, as well as generating the results.

AI tools did not substitute for any analytical reasoning or experimental work — they provided support for interpreting and presenting the results with human oversight.
